\documentclass[a4paper, 12pt]{article}
\usepackage[a4paper,top=1.5cm, bottom=1.5cm, left=1cm, right=1cm]{geometry}
\usepackage[utf8]{inputenc}
\usepackage{mathtext}
\usepackage{amsmath}
\usepackage{amssymb}
\usepackage{amsfonts}
\usepackage{textcomp}
\usepackage[english, russian]{babel}
\usepackage{indentfirst}
\usepackage{fixltx2e}
\usepackage{longtable}
\usepackage{graphicx}
\graphicspath{{pictures/}}
\DeclareGraphicsExtensions{.pdf,.png,.jpg}
\usepackage{natbib}
\usepackage{mathrsfs}
\usepackage[europeanresistors, americaninductors]{circuitikz}

\title{АЛКТГ дз 2}
\author{Татаринов Георгий}
\date{\today}

\begin{document}
	\maketitleј
	\section*{Задание 1}
		Имеется набор из 6 человек, нужно заполнить 6 вакансий жтими людьми. Тогда количество способов равно числу размещений 6 человек по 6 вакансиям, а именно $A_6^6=6!=720$
		Ответ: 720
	\section*{Задание 2а}
		Всего чисел от 0 до 1000000: 1000001. Посчитаем количество чисел, не содержащих единиц. Число 1000000 содержит 1, а значит, оно не подходит, а подходят только числа не более чем пятизначные. Считаем, что все числа пятизначные, для этого добавим, где нужно сколько нужно ведущих нулей. Тогда каждое из таких чисел можно получить, поставив на позиции цифры от 0 до 9 без 1, причём любое число, собранное таким способом будет подходить. Таких чисел $9**5=59045$. Числа, содержащие 1, это все остальные числа, их $1000001-59045=940952$. Их больше.
		Ответ: больше тех, в записи которых есть 1.
	\section*{Задание 2б}
		Всего чисел от 0 до 10000000: 10000001. Посчитаем количество чисел, не содержащих единиц. Число 10000000 содержит 1, а значит, оно не подходит, а подходят только числа не более чем шестизначные. Считаем, что все числа шестизначные, для этого добавим, где нужно сколько нужно ведущих нулей. Тогда каждое из таких чисел можно получить, поставив на позиции цифры от 0 до 9 без 1, причём любое число, собранное таким способом будет подходить. Таких чисел $9\mathaccent6=531441$. Числа, содержащие 1, это все остальные числа, их $10000001-531441=9468560$. Их больше.
		Ответ: больше тех, в записи которых есть 1.
	\section*{Задание 3}
		Посчитаем количество шестизначных чисел. Каждое число можно получить выбрав на 1 место 1 цифру из 9, а на каждое из остальных - из 10. Всего чисел 900000.
		Посчитаем количество шестизначных чисел, в которых нет одинаковых цифр, но может быть ведущие нули. Каждое число можно получить, расместив 10 цифр на 6 позиций без повторений, то есть способов $A_10^6=151200$.
		Посчитаем количество шестизначных чисел, в которых нет одинаковых цифр, и обязательно есть ведущие нули. Первая цифра известна, а по остальным, расмещаем 10 цифр на 5 позиций без повторений, то есть способов $A_9^5=15120$.
		Посчитаем количество шестизначных чисел, в которых нет одинаковых цифр. Для этого вычтем из количества шестизначных чисел без повторений с возможными ведущими нулями количество шестизначных чисел с ведущими нулями. Получим 27216.
		Посчитаем количество шестизначных чисел, в которых есть одинаковые цифры. Для этого вычтем из количества шестизначных чисел количество шестизначных чисел без повторов цифр. Получим 872784.
		Посчитаем Вероятность, что любое шестизначное число будет одним из шестизначных чисел, в которых есть одинаковые цифры. Для этого поделим количество шестизначных чисел с повторениями цифр на общее количество шечтизначных чисел. Получим $6061/6250$
		Ответ: $6061/6250$
	\section*{Задание 4}
	\section*{Задание 6}
		Чило $x$ из множества \{1,2,3,4,...,n\} может быть отображено на любое из чисел \{x,x+1,...,m\}. Тогда количество способов выбрать, куда отображать $x$ равно max(0,m-x+1). При этом, выбор отбражений каждого из чисел не зависит от выбора для других чисел. Тогда всего существует отображений max(0,m-1+1)*max(0,m-2+1)*max(0,m-3+1)*...*max(0,m-n+1). При этом, если какой-то множитель стал равен 0, то значение, передаваемое в правый аргумент max стало меньше либо равно 0, то так как в каждом множителе значение, передаваемое в правый аргумент max на 1 меньше, чем в множителе слева, и все значения целые, и значение в первом множителе равно m>0, то обязательно найдётся множитель, у которого правое значение равно 0. Тогда количество отображений равно (m)*(m-1)*...*(m-n+1).
		Ответ: m*(m-1)*...*(m-n+1)
	\section*{Задание 7}
		Выберем студента в одноместную комнату. Это 7 способов. Из оставшихся выберем 2 студента в двуместную комнату, это $C_6^2=15$. Остальных в четырёхместную. Итого, $7*15=105$ способов.
		Ответ: 105 способов.
	\section*{Задание 8}
	\section*{Задание 9}

	\section*{Задание 10}
		Каждой скобочной последовательности сопоставим последовательность из чисел 1 и -1, где открывающей скобке ставится в соответствие 1, закрывающей -- -1. Заметим, что получилась биекция между скобочными последовательностями и последовательностями из чисел -1 и 1. Каждой скобочной последовательности поставлена в соответствие последовательность из 1 и -1, в которой сумма всех членов равна 0, так как в правильной скобочной последовательности количества открывающих и закрывающих скобок совпадают. Однако, существуют последовательности из -1 и 1, сумма которых 0, однако, соответствующие им скобочные последовательности не являются правильными, например, последовательности -1,1 соответствует последовательность )(.
		Ответ: правильных скобочных последовательностей меньше.
\end{document}
