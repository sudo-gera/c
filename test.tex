\documentclass[a4paper, 12pt]{article}
\usepackage[a4paper,top=1.5cm, bottom=1.5cm, left=1cm, right=1cm]{geometry}
\usepackage[utf8]{inputenc}
\usepackage{mathtext}
\usepackage{amsmath}
\usepackage{amssymb}
\usepackage{amsfonts}
\usepackage{textcomp}
\usepackage[english, russian]{babel}
\usepackage{indentfirst}
\usepackage{fixltx2e}
\usepackage{longtable}
\usepackage{graphicx}
\usepackage{tabto}
\graphicspath{{pictures/}}
\DeclareGraphicsExtensions{.pdf,.png,.jpg}
\usepackage{natbib}
\usepackage{mathrsfs}
\usepackage[europeanresistors, americaninductors]{circuitikz}
\renewcommand{\t}[1]{\hspace{#1mm}}
\newcommand{\n}{\par{}}

\title{АЛКТГ дз 2}
\author{Татаринов Георгий}
\date{\today}

\begin{document}
	\maketitle
	\section*{Задание 1}
		Рассмотрим таблицу, в которой в ячейке с координатами x,y запишем количество способов добраться до неё из (0,0).\n
		1\t{9}1\t{9}1\t{9}1\t{9}1\n
		1\t{9}2\t{9}3\t{9}4\t{9}5\n
		1\t{9}3\t{9}7\t{9}12\t{7}18\n
		1\t{9}4\t{9}12\t{7}26\t{7}47\n
		1\t{9}5\t{9}18\t{7}47\t{7}101\n
		1\t{9}6\t{9}25\t{7}76\t{7}189\n
		Ответ: 189
	\section*{Задание 2}
		Количество способов купить 100 пирожков из 10 возможных равно количеству способов вставить 9 перегородок между 100 пирожками, что равно количеству способов выбрать среди 109 пирожков и перегородок 9 перегородок, что равно $C_{109}^9$.
		Ответ: $C_{109}^9$
	\section*{Задание 3}
		% $(1+2)^n=\Sigma_{k=0}^n(C_n^k*2^n)$
	\section*{Задание 4}
	\section*{Задание 5}
	\section*{Задание 6}
		$C_{F_{1000}}^{F_{998}+1}=C_{F_{1000}}^{F_{998}}*(F_{998}+1)/(F_{1000}-F_{998})=C_{F_{1000}}^{F_{998}}*(F_{998}+1)/(F_{999})$ \n
		$C_{F_{1000}}^{F_{999}+1}=C_{F_{1000}}^{F_{999}}*(F_{999}+1)/(F_{1000}-F_{999})=C_{F_{1000}}^{F_{999}}*(F_{999}+1)/(F_{998})$ \n
		$C_{F_{1000}}^{F_{999}}=C_{F_{1000}}^{F_{998}}$\n
		$(F_{999}+1)/(F_{998})>(F_{998}+1)/(F_{999})$ \n
		Ответ: $C_{F_{1000}}^{F_{999}+1}>C_{F_{1000}}^{F_{998}+1}$
	\section*{Задание 7}
	\section*{Задание 8}
	\section*{Задание 9}
		Результат - разложение числа 8 в сумму 8 неотрицательных слагаемых, их количество равно количеству способов расставить 7 пеергородок между 8 шарами, что равно $C_{15}^7=6435$
		Ответ: 6435
	\section*{Задание 10}
\end{document}
