\documentclass[a4paper, 12pt]{article}
\usepackage[a4paper,top=1.5cm, bottom=1.5cm, left=1cm, right=1cm]{geometry}
\usepackage[utf8]{inputenc}
\usepackage{mathtext}
\usepackage{amsmath}
\usepackage{amssymb}
\usepackage{amsfonts}
\usepackage{textcomp}
\usepackage[english, russian]{babel}
\usepackage{indentfirst}
\usepackage{fixltx2e}
\usepackage{longtable}
\usepackage{graphicx}
\usepackage{tabto}
\graphicspath{{pictures/}}
\DeclareGraphicsExtensions{.pdf,.png,.jpg}
\usepackage{natbib}
\usepackage{mathrsfs}
\usepackage[europeanresistors, americaninductors]{circuitikz}
\renewcommand{\t}[1]{\hspace{#1mm}}
\newcommand{\n}{\par{}}
\renewcommand{\d}{$}
\renewcommand{\a}{\bigcap{}}
\renewcommand{\u}{\bigcup{}}
\newcommand{\png}[1]{\includegraphics[width=200pt]{~/Documents/#1.png}\n}
\renewcommand{\i}{^{-1}}

\title{АЛКТГ дз 2}
\author{Татаринов Георгий}
\date{\today}

\begin{document}
	\maketitle
	\section*{Задание 1}
		$R \subseteq \{1, 2, 3\} \times \{1, 2, 3\}$\n
		а)$R = \{(1, 1), (2, 2), (3, 3), (1, 2), (1, 3), (3, 2)\}$\n
		\png{1}
		R рефлексивно, так как для всех a из \{1, 2, 3\} $(a, a) принадлежит R$

		R не симметрично, так как (1, 2) принадлежит R, а (2, 1) не принадлежит R

		R транзитивно, так как R х R - имеет тот же граф, что и R

		R не отношение эквивалентноти, так так R не симметрично


		б)$R = \{(1, 1), (1, 2), (2, 1), (2, 2)\}$

		\png{2}

		R не рефлексивно, так как (3, 3) не принадлежит R

		R симметрично, так как граф неориентированный

		R транзитивно, так как R х R - имеет тот же граф, что и R

		R не рефлексивно следовательно, R не отношение эквивалентноти

	\section*{Задание 2}
		пусть M - отношение быть матерью
		
		пусть P - отношение быть отцом

		пусть R - отношение быть племянником

		тогда: \d R=M \i \circ M \i \circ M \u M \i \circ P \i \circ P \u P \i \circ M \i \circ M \u P \i \circ P \i \circ P \d
	\section*{Задание 3}
		пусть А=\{1,2,3,4,5\}

		$P_1=\{(1,2),(3,4)\}$

		$P_2=\{(2,3),(4,5)\}$

		тогда:

		$\overline{P_1}$ содержит (1,3) и (3,2), но не содержит (1,2) значит не транзитивно

		$P_1\u P_2$ содержит (1,2) и (2,3) но не содержит (1,3) значит не транзитивно

		$P_1\circ P_2$ содержит (1,3) и (3,5) но не содержит (1,5) значит не транзитивно

		для любого A:

		$P_1\u P_2$ транзитивно, так как если $P_1\u P_2$ соержит (a,b) и (b,c) то каждое из $P_1$ и $P_2$ содержит (a,b) и (b,c), тогда каждое из $P_1$ и $P_2$ содержит (a,c) тогда $P_1\u P_2$ содержит (a,c)

	\section*{Задание 4}
		Пусть множество \{1,2,3,4,5,6\}
		
		Рассмотрим бинарное отношение из любого элемента в любой, кроме отношений (4,4),(5,5),(6,6)

		тогда это отношеие симметрично и состоит из 33 пар

		а) ответ: да

		если из множества всех пар выбросить пару (1,1), то придётся выбросить ещё по одной из пар (1,2) и (2,1), (1,3) и (3,1), (1,4) и (4,1), (1,5) и (5,1), (1,6) и (6,1). Тогда так как всего пар 36, останется не более 30.

		если из множества всех пар выбросить пару (1,2), то придётся выбросить ещё по одной из пар (1,3) и (3,2), (1,4) и (4,2), (1,5) и (5,2), (1,6) и (6,2). Тогда так как всего пар 36, останется не более 31.

		случаи выбрасывания остальных пар аналогичны этим

		б) ответ: нет
	\section*{Задание 5}
	\section*{Задание 6}
	\section*{Задание 7}
	\section*{Задание 8}
	\section*{Задание 9}
	\section*{Задание 10}
\end{document}
