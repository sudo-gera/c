\documentclass[a4paper, 12pt]{article}
\usepackage[a4paper,top=1.5cm, bottom=1.5cm, left=1cm, right=1cm]{geometry}
\usepackage[utf8]{inputenc}
\usepackage{mathtext}
\usepackage{amsmath}
\usepackage{amssymb}
\usepackage{amsfonts}
\usepackage{textcomp}
\usepackage[english, russian]{babel}
\usepackage{indentfirst}
\usepackage{fixltx2e}
\usepackage{longtable}
\usepackage{graphicx}
\graphicspath{{pictures/}}
\DeclareGraphicsExtensions{.pdf,.png,.jpg}
\usepackage{natbib}
\usepackage{mathrsfs}
\usepackage[europeanresistors, americaninductors]{circuitikz}

\title{АЛКТГ дз 2}
\author{Татаринов Георгий}
\date{\today}

\begin{document}
	\maketitleј
	\section*{Задание 1}
		$Dom(h)={1,2,3,4,5,6,8}$\par
		$Range(h)={b,c,e,f}$\par
		$h({0,1,2,3,4})={b,c,e}$\par
		$h^-1({a,b,c})={1,2,3,5,8}$\par
		$h^-1(h({1,2,3,4,5,6,8})$\par
		$h(h^-1({a,b,c,d,e}))={b,c,e}$\par
	\section*{Задание 2}
		квадрат целого числа всегда не меньше, чем модуль самого числа, f ставит в соответсивие каждому целому числу простое число, не меньшее его квадрата, а значит и не меньшее его модуля. Тогда если множество X конечно, то модуль каждого прообраза каждого элемента X не больше самого элемента. Тогда так как множество X конечно, то в нём можно выбрать максимальный элемент и модуль каждого элемента прообраза будет не больше, чем максимальный элемент X. Тогда прообраз X конечен.
	\section*{Задание 3}
		Пусть X={1,2,3} A={1,2} Y={2}. f: 2->2, 3->2.
		Тогда f(A)={2}
		$f^-1(f(A))={2,3}$
		Ответ: никакой из знаков не подходит в $f^-1(f(A))?A$
	\section*{Задание 4}
		Пусть A={1}, B={2}, Y={1}, f: 1->1, 2->1
		тогда
		$A\setminus B={1}$
		$f(A\setminus B)={1}$
		f(A)={1}
		f(B)={1}
		$f(A)\setminus f(B)={}$
		Если $y$ принадлежит $f(A)\setminus f(B)$, то в A существует $x$, что f(x)=y, при этом, в B не существует элемента, f от которого $y$, в том числе, $x$ не принадлежит B. отсюда следует, что $x$ принадлежит $A\setminus B$, а значит, $y$ принаждежит $f(A\setminus B)$
	\section*{Задание 6}
		Рассмотрим путь из 3 вершин. В нём нечётное количество вершин, поэтому любое паросочетание не является совершенным
		Ответ: неверно
	\section*{Задание 7}
		Пусть B={1}, Y={1,2}, X={1}, f: 1->1
		тогда $f^-1$(B)=$X$
		Ответ: неверно
	\section*{Задание 8}
		Пусть B={1}, Y={1}, X={1}, f: 1->1
		тогда $B\neq{}$ и $f^-1$$(B)\neq {}$
	\section*{Задание 9}
		Каждому множеству сопоставим последовательность, которая получается из него упорядочиванием элементов по возрастанию. Разным множествам сопоставлены разные последовательности, так как в разных множествах набо элементов разный и соответственно, в сопоставленных им последовательностиях набор элементов раный. Каждая последовательность сопоставлена множеству, которое состоит из её элементов. Значит, данное отображение является биекцией.
\end{document}
