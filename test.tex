\documentclass[a4paper, 12pt]{article}
\usepackage[a4paper,top=1.5cm, bottom=1.5cm, left=1cm, right=1cm]{geometry}
\usepackage[utf8]{inputenc}
\usepackage{mathtext}
\usepackage{amsmath}
\usepackage{amssymb}
\usepackage{amsfonts}
\usepackage{textcomp}
\usepackage[english, russian]{babel}
\usepackage{indentfirst}
\usepackage{fixltx2e}
\usepackage{longtable}
\usepackage{graphicx}
\usepackage{tabto}
\graphicspath{{pictures/}}
\DeclareGraphicsExtensions{.pdf,.png,.jpg}
\usepackage{natbib}
\usepackage{mathrsfs}
\usepackage[europeanresistors, americaninductors]{circuitikz}
\renewcommand{\t}[1]{\hspace{#1mm}}
\newcommand{\n}{\par{}}
\renewcommand{\d}{$}
\renewcommand{\a}{\bigcap{}}
\renewcommand{\u}{\bigcup{}}

\title{АЛКТГ дз 2}
\author{Татаринов Георгий}
\date{\today}

\begin{document}
	\maketitle
	\section*{Задание 1}
		пусть:\n
		\t{11}\d A\d - множество всех раскрасок\n
		\t{11}\d A_1\d - множество всех раскрасок, где незакрашенные клетки содержат верхний ряд\n
		\t{11}\d A_2\d - множество всех раскрасок, где незакрашенные клетки содержат две средних вертикали\n
		\t{11}\d A_3\d - множество всех раскрасок, где незакрашенные клетки содержат нижний ряд\n
		тогда:\n
		\t{11}\d |A|=2^{12} \d\n
		\t{11}\d |A_1|=2^{8} \d\n
		\t{11}\d |A_2|=2^{6} \d\n
		\t{11}\d |A_3|=2^{8} \d\n
		\t{11}\d |A_1\bigcap A_2|=2^{4} \d\n
		\t{11}\d |A_1\bigcap A_3|=2^{4} \d\n
		\t{11}\d |A_2\bigcap A_3|=2^{4} \d\n
		\t{11}\d |A_1\bigcap A_2\bigcap A_3|=2^{2} \d\n
		\d |A_1\bigcup A_2\bigcup A_3|=|A_1|+|A_2|+|A_3|-|A_1\bigcap A_2|-|A_3\bigcap A_1|-|A_2\bigcap A_3|+|A_1\bigcap A_2\bigcap A_3|=8+6+8-4-4-4+2=12\d\n
		Ответ:12
	\section*{Задание 2}
		Рассмотрим двудольный граф, в одной доле которого профессии, в другой - люди, вершины соединены ребром, если человек владеет профессией. Тогда степень каждой вершины профессии равна 6, что озночает, что всего рёбер 24. Каждой парой профессий владеет по 4 человека, это значит, что из каждой пары профессий, а их 6, можем найти 4 человека, каждый из которых соединён только с этими 2 профессиями, и ни с какими более, что означает, что можем найти 6*4 человека, степень каждого 2. Их суммарная степень равна 48>24. Противоречие.\n
		Ответ: невыполнимо
	\section*{Задание 3}
		Если |B|>0 и |A|>0, то в B есть хотя бы 1 элемент. Отобразим все элементы A в этот элемент и получим сюрьекцию. Тогда существует хотя бы 1 сюрьекция, что значит, что количество элементов в B не более, чем в A. Ещё существует хотя бы 1 инъекция, что значит, что количество элементов в B не менее, чем в A. Тогда все сюрьекции - биекции. Их количество n!
		Если |B|=0 и |A|=0, то 0!=1
		Ответ: n!
	\section*{Задание 4}
	\section*{Задание 5}
	\section*{Задание 6}
	\section*{Задание 7}
	\section*{Задание 8}
	\section*{Задание 9}
	\section*{Задание 10}
\end{document}
