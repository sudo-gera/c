\documentclass[a4paper, 12pt]{article}
\usepackage[a4paper,top=1.5cm, bottom=1.5cm, left=1cm, right=1cm]{geometry}
\usepackage[utf8]{inputenc}
\usepackage{mathtext}
\usepackage{amsmath}
\usepackage{amssymb}
\usepackage{amsfonts}
\usepackage{textcomp}
\usepackage[english, russian]{babel}
\usepackage{indentfirst}
\usepackage{fixltx2e}
\usepackage{longtable}
\usepackage{graphicx}
\graphicspath{{pictures/}}
\DeclareGraphicsExtensions{.pdf,.png,.jpg}
\usepackage{natbib}
\usepackage{mathrsfs}
\usepackage[europeanresistors, americaninductors]{circuitikz}

\title{АЛКТГ дз 2}
\author{Татаринов Георгий}
\date{\today}

\begin{document}
	\maketitleј
	\section*{Задание 1}
		Рассмотрим граф-цикл длины 3. В нём степень каждой вершины равна 2. Предположим, что он 2 раскрашиваемый. Тогда среди 2 вершин найдётся 2 одного цвета, при этом в цикле длины 3 любые 2 вершины сосоедние. Тогда раскраска неправильная. Противоречие.
		Ответ: неверно
	\section*{Задание 2}
		Дерево на $2n$ вершинах имеет 2-расскраску, при этом, вершин хотя бы одного из 2 цвтеов хотя бы $n$. Выберем $n$ вершин одного цвета. Так как раскраска правильная, среди них нет 2 вершин, соединённых ребром, то есть раскраска правильная.
	\section*{Задание 3}
		Пусть в дереве на 2021 вершинах 3 имеют степень 1, $x$ имеют степень 3, $y$ имеют степень хотя бы 4, $2018-x-y$ имеют степень 2.
		Тогда сумма всех степеней равна удвоенному количеству рёбер и равна 4040. Тоесть, $3+(2018-x-y)*2+x*3+y*4\leq4040$, из чего следует, что $x+2*y\leq1$, отсюда, если $y\geq1$, то $x+2*y\geq2$ - противоречие, тогда $y=0$, то есть, $3+(2018-x-y)*2+x*3=4040$, то есть, $x=1$.
		Ответ: 1
	\section*{Задание 4}
		Пусть существуют такие 2 деерва. Так как диаметр каждого из них рааен $d$, то в каждом из них есть путь длины $d$. Добавим между ними ребро так, чтобы диаметр полученного графа стал равен $d$. Для каждой из двух половин этого графа рассмотрим пути от принадлежащей этой половине графа вершины этого ребра до концов путей длины $d$. Каждый из этих путей состоит из пути от вершиныи общего ребра до пути длины $d$ и из части пути длины $d$. Тогда сумма длин этих двух путей до концов пути длины $d$ больше либо равна $d$, так как состоит из длины пути $d$ и удвоенной длины до этого пути, что значит, что хотя бы один из двух путей не короче чем $d/2$. Возьмём эти пути для каждой из двух половин графа и ребро между половинами. Полученный путь будет иметь длину $d+1$ - противоречие.
		Ответ: не сучествует.
	\section*{Задание 5}
		Будем красить вершины по очереди. На каждой вершине смотрим, какие соседи уже раскрашены, смотрим, какие там цвета, так как соседей не более $d$, то они раскрашены не больее, чем в $d$ цветов, то есть есть хотя бы 1 цвет, в который не покрашен никто из соседей. В этот цвет красим рассматриваемую вершину. Если до покраски вершины не было ребра с вершинами одного цвета, то и после покраски вершины не будет. Изначально их небыло, значит, и в конце их не будет и раскраска будет правильной.
	\section*{Задание 8А}
		Если $n$ делится на 2, то при соединении противоположных вершин образуется цикл $v_1$, $v_2$, ..., $v_n_1$, $v_1$ длины $n+1$, то есть цикл нечётной длины. Из чего следует, что граф не 2-раскрашиваемый.\par
		Если $n$ не делится на 2, то раскрасим граф по кругу в 2 цвета чередуя цвета. Рёбра, соединяющие противоположные вершины будут соединять вершины разного цвета и раскраска будет правильной.
		Ответ: при $n$ не делится на 2.
	\section*{Задание 8А}
		Если $n$ делится на 2, то раскрасим граф по кругу в 2 цвета чередуя цвета. После этого, поменяем цвета у всех вершин с номерами большими $n$, а вершины $n$ и $2n$. Рёбра, соединяющие противоположные вершины будут соединять вершины разного цвета и раскраска будет правильной.
		Если $n$ не делится на 2, то раскрасим граф по кругу в 2 цвета чередуя цвета. Рёбра, соединяющие противоположные вершины будут соединять вершины разного цвета и раскраска будет правильной.
		Ответ: при $n$ не делится на 2.
\end{document}
