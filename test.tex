\documentclass[a4paper, 12pt]{article}
\usepackage[a4paper,top=1.5cm, bottom=1.5cm, left=1cm, right=1cm]{geometry}
\usepackage[utf8]{inputenc}
\usepackage{mathtext}
\usepackage{amsmath}
\usepackage{amssymb}
\usepackage{amsfonts}
\usepackage{textcomp}
\usepackage[english, russian]{babel}
\usepackage{indentfirst}
\usepackage{fixltx2e}
\usepackage{longtable}
\usepackage{graphicx}
\usepackage{tabto}
\graphicspath{{pictures/}}
\DeclareGraphicsExtensions{.pdf,.png,.jpg}
\usepackage{natbib}
\usepackage{mathrsfs}
\usepackage[europeanresistors, americaninductors]{circuitikz}
\renewcommand{\t}[1]{\hspace{#1mm}}
\newcommand{\n}{\par{}}
\renewcommand{\d}{$}
\renewcommand{\a}{\bigcap{}}
\renewcommand{\u}{\bigcup{}}
\newcommand{\png}[1]{\includegraphics[width=200pt]{~/Documents/#1.png}\n}
\renewcommand{\i}{^{-1}}

\title{АЛКТГ дз 2}
\author{Татаринов Георгий}
\date{\today}

\begin{document}
	\maketitle
	\section*{Задание 1}
		Расмотрим граф, в котором есть вершины \{1,2,3,4\} и рёбра \{1,2\},\{1,3\},\{1,4\},\{2,3\},\{3,4\}

		тогда циклический маршрут по всем рёбрам по 2 раза это \{1,2\}, \{2,3\}, \{3,4\}, \{4,1\}, \{1,2\}, \{2,3\}, \{3,4\}, \{4,1\}, \{1,3\}, \{3,1\}

		При этом, если существует эйлеров маршрут, рассмотрим произвольную вершину. По каждому ребру, соединённому с вершиной маршурт либо входит в вершину, либо выходит из неё. При этом, количество входов и выходов одинаково. Тогда степерь каждой вершины чётна, а в данном графе это не так.
	\section*{Задание 2}
		Пусть в графе есть более 1 компоненты связности. Рассмотрим граф, где вершины это компоненты сильной связности исходного. В нём нет циклов, значит, есть сток. В этой компоненте сильной связности рассмотрим произвольную вершину. Все рёбра, исходящие из этой вершины остаются в этой компоненте связности. Тогда там не менее n-2 вершин. Тогда всего компонет связности не более 2. Расмотрим граф, в котором все вершины - остатки по модулю n, из вершины k идут рёбра в врешины k+1,...,k+n-2. Тогда этот грав сильносвязный и в нём 1 компонента сильной свзяности. Рассмотрим граф на n-1 вершине, где каждая пара вершин соединена 2 рёюрами в обе стороны, добавим вершину. из которой будут идти рёбра в n-2 существующих, но не будут в неё. Тогда в графе будет исходящая степерь каждой вершины равна n-2, и 2 компонеты сильной связности.

		Ответ: 2
	\section*{Задание 3}
		Пусть \d A_1,...,A_n \d - самый длинный простой путь. Пусть есть вершина B которая этому пути не принадлежит. Пуcть k - максимальное число из \{1,...,n+1\}такое, что для всех 0<j<k есть ребро $A_jB$. Тогда если 1<k<n+1, то есть ребро $A_{k-1}B$ и ребро $BA_{k}$, если k=1, то есть ребро $BA_0$, если k=n+1, то есть ребро $A_nB$. Во всех случаях, вершина B может быть вставлена в путь, значит, путь не самый длинный. Тогда приедположение о существовании точки вне пути неверно.
	\section*{Задание 4}
		очки $<_P$ брюки $<_P$ ремень $<_P$ рубашка $<_P$ галстук $<_P$ пиджак $<_P$ нсоки $<_P$ туфли $<_P$ часы
	\section*{Задание 5}
		Пронумеруем города произвольным образом. каждое ребро ориентируем так, чтобы оно вело их большего города в меньший. Тогда перемещаясь по рёюрам будем всегда перемещаться от городоа с большим номером к меньшему, значит, никогда не вернёмся в город с исходным номером.

		Если есть цикл. то выехав их каждого города и проехав по циклу в него можно вернуться. Тогда циклов нет. В ориентироваом графе без циклов есть сток и исток. Из стока нельзя выехать, для любого города, кроме истока, выполнено, что между ним и истоком есть ребро и оно направлено из истока. Тогда из истока можно добраться до любого города.

		Пусть \d A_1,...,A_n \d - самый длинный простой путь. Пусть есть вершина B которая этому пути не принадлежит. Пуcть k - максимальное число из \{1,...,n+1\}такое, что для всех 0<j<k есть ребро $A_jB$. Тогда если 1<k<n+1, то есть ребро $A_{k-1}B$ и ребро $BA_{k}$, если k=1, то есть ребро $BA_0$, если k=n+1, то есть ребро $A_nB$. Во всех случаях, вершина B может быть вставлена в путь, значит, путь не самый длинный. Тогда приедположение о существовании точки вне пути неверно. Тогда существует путь, проходящий по всем вершинам. Пронумеруем вершины в том порядке. в котором они в этом пути. Пусть есть ещё один путь. Тогда так ак он отличается от того пути, то в нём есть 2 вершины с номерами C и D, причём C<D и D встречается раньше, чем С. Тогда из первого пути возьмём путь из C в D, а из второго из D в C. Тогда можно выйти из C и вернуться в C - проиворечие с предположением о существовании второго пути.

		В кажодом способе есть единственный путь, проходящий по всем верщинам. Пронумеруем вершины в том порядке, в каком они в пути. Тогда для любых номеров C и D, где C<D, из пути по всем вершинам можно взять путь из C в D. Тогда ребро между ними направлено из C в D, так как иначе можно будет выйти из С, по пути дойти до D, а потом по ребру вернуться. Тогда каждый способ однозначно определяется способом пронумеровать города. 

		Ответ: n!
	\section*{Задание 6}
		Пусть в графе нет a, b, c, что aPb, bPc и cPa. Тогда любые 3 вершины графа это строгий линейный порядок. Докажем по индукции, что любые k врешин это строгий линейный порядок. База для k=1,2,3 уже есть. Переход k-1 -> k: среди k вершин выберем вершину B, а остальные между собой образуют строгий линейный порядок. Пусть это $A_1,...,A_{k-1}$ причём нумерация в том порядке, в котором задан линейный порядок. Тогда пусть o - максимальный номер вершины из $\{A_1,...,A_{k-1}\}$, из которой есть ребро в B, p - минимальный номер вершины из $\{A_1,...,A_{k-1}\}$ в которую есть путь из B. Если $o>p$ то есть путь из $A_o$ в B, из B в $A_p$, из $A_p$ в $A_o$, - противоречие.
		Тогда o<p. При этом для всех $A_{o+1},...,A_{p-1}$ не может быть ребра ни в B ни из B. Это возможно только если o+1=p. Тогда есть рёбра из B во все врешины $A_{o+1}$,...,$A_{k-1}$, так как из них не может быть ребра в B, есть рёбра в B изо всех врешин $A_{1},...,A_{p-1}$, так как в них не может быть ребра из B. Тогда множество $\{B,A_1,...,A_k\}$ является строгим линейным порядком. Тогда и весь турнир - строгий линейный порядок.
	\section*{Задание 7}
		выберем пару, которая будет несравнима (n способов), объявим акое-либо сравнение на оставшихся парах ($2^{C_n^2-1}$ способов). Всего способов $n*2^{C_n^2-1}$

		Ответ: $n*2^{C_n^2-1}$
	\section*{Задание 8}
	\section*{Задание 9}
	\section*{Задание 10}
\end{document}
