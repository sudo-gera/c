\documentclass[a4paper, 12pt]{article}
\usepackage[a4paper,top=1.5cm, bottom=1.5cm, left=1cm, right=1cm]{geometry}
\usepackage[utf8]{inputenc}
\usepackage{mathtext}
\usepackage{amsmath}
\usepackage{amssymb}
\usepackage{amsfonts}
\usepackage{textcomp}
\usepackage[english, russian]{babel}
\usepackage{indentfirst}
\usepackage{fixltx2e}
\usepackage{longtable}
\usepackage{graphicx}
\usepackage{tabto}
\graphicspath{{pictures/}}
\DeclareGraphicsExtensions{.pdf,.png,.jpg}
\usepackage{natbib}
\usepackage{mathrsfs}
\usepackage[europeanresistors, americaninductors]{circuitikz}
\renewcommand{\t}[1]{\hspace{#1mm}}
\newcommand{\n}{\par{}}
\renewcommand{\a}{\bigcap{}}
\renewcommand{\u}{\bigcup{}}
\newcommand{\png}[1]{\includegraphics[width=200pt]{~/Documents/#1.png}\n}
\renewcommand{\d}{$}
\renewcommand{\i}{^{-1}}
\renewcommand{\d}{$}
\newcommand{\dand}{$\land$}
\newcommand{\dor}{$\lor$}
\newcommand{\dno}[1]{$\overline{#1}$}
\newcommand{\dnot}{$\neg$}



\title{АЛКТГ дз 2}
\author{Татаринов Георгий}
\date{\today}

\begin{document}
	\maketitle
	\section*{Задание 1}

		ДНФ: $ \overline{x_1} x_2 \overline{x_3} \vee \overline{x_1} x_2 x_3 \vee x_1 \overline{x_2} \overline{x_3} \vee x_1 \overline{x_2} x_3$

		КНФ: $ (x_1 \vee x_2 \vee x_3)(x_1 \vee x_2 \vee \overline{x_3})(\overline{x_1} \vee \overline{x_2} \vee x_3)(\overline{x_1} \vee \overline{x_2} \vee \overline{x_3})$

	\section*{Задание 2}
		$\neg (1->1)=\neg 1 =0$

		Ответ: да
	\section*{Задание 3}
		MAJ(x, y, z) = xy+xz+yz

		функция симметрична относительно x,y,z

		MAJ(0,0,0)=0

		MAJ(1,0,0)=0

		MAJ(1,1,0)=0

		MAJ(1,1,1)=0
	\section*{Задание 4}
		x_1 \dor x_2 \dor ... x_n=x_1+x_2+...+x_n+x_1x_2+...+x_1x_2...x_n

		количество слагаемых - все способы выбрать непустое подмножетсво множества переменных.

		Ответ: 2^n-1
	\section*{Задание 5}
		\dnot(x\dand x)=\dnot x

		\dnot \dnot(x \dand y) = x \dand y

		\dnot (\dnot x \dand \dnot y) = x \dor y
	\section*{Задание 6}
		1->1=1

		1\dor1=1

		базис содержится в T_1

		Ответ: нет
	\section*{Задание 7}
		$\lnot \overline{x}$ = $\overline{\lnot x}$
		
		\dnot MAJ(x_1,x_2,x_3) = MAJ(\dnot x_1,\dnot x_2,\dnot x_3)

		базис содержится в S

		Ответ: нет
	\section*{Задание 8}
		f - не монотонная -> есть переменная x_i и набор значений a_j из \{0,1\} для остальных переменных, что f(a_1,a_2,...,a_{i-1},1,...,a_n)=0 и f(a_1,a_2,...,0,...,a_n)=1

		тогда g(x)=f(a_1,a_2,...,a_{i-1},x,...,a_n)
	\section*{Задание 9}
		Представим нашу функцию $f(x_1, x_2, ..., x_n)$ в дизъюнктивной нормальной форме, тогда если в каком-то слагаемом был элемент $\neg x_i$, то в силу монотонности есть слагаемое, где есть элемент x_1, а остальные множители такие же. Объединим их и избавимся от $\neg x_i$. И так пока не придём к базису.
	\section*{Задание 10}
		а)

		PAR(x_1,x_2,...,x_n)=x_1+x_2+...+x_n

		индукция по n

		база n=1

		PAR (x_1)=x_1 верно

		переход x_{n-1} -> x_n

		PAR(x_1,...,x_{n-1},0)=PAR(x_1,...,x_{n-1})=PAR(x_1,...,x_{n-1})+0

		PAR(x_1,...,x_{n-1},1)=\dnot PAR(x_1,...,x_{n-1})=PAR(x_1,...,x_{n-1})+1

		б)

		при  n=1

		PAR (x_1)=x_1

		при n>1

		PAR(1,1,0,...,0)=0

		PAR(1,0,0,...,0)=1

		функция PAR не монотонна -> не представима ввиде ДНФ без отрицаний

		Ответ: при n=1
\end{document}
