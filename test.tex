\documentclass[a4paper, 12pt]{article}
\usepackage[a4paper,top=1.5cm, bottom=1.5cm, left=1cm, right=1cm]{geometry}
\usepackage[utf8]{inputenc}
\usepackage{mathtext}
\usepackage{amsmath}
\usepackage{amssymb}
\usepackage{amsfonts}
\usepackage{textcomp}
\usepackage[english, russian]{babel}
\usepackage{indentfirst}
\usepackage{fixltx2e}
\usepackage{longtable}
\usepackage{graphicx}
\graphicspath{{pictures/}}
\DeclareGraphicsExtensions{.pdf,.png,.jpg}
\usepackage{natbib}
\usepackage{mathrsfs}
\usepackage[europeanresistors, americaninductors]{circuitikz}

\title{АЛКТГ дз 2}
\author{Татаринов Георгий}
\date{\today}

\begin{document}
	\maketitleј
	\section*{Задание 1}
		Dom(h)={1,2,3,4,5,6,8}
		Range(h)={b,c,e,f}
		h({0,1,2,3,4})={b,c,e}
		h^-1({a,b,c})={1,2,3,5,8}
		h^-1(h({1,2,3,4,5,6,8})
		h(h^-1({a,b,c,d,e}))={b,c,e}
	\section*{Задание 2}
		квадрат целого числа всегда не меньше, чем модуль самого числа, f ставит в соответсивие каждому целому числу простое число, не меньшее его квадрата, а значит и не меньшее его модуля. Тогда если множество X конечно, то модуль каждого прообраза каждого элемента X не больше самого элемента. Тогда так как множество X конечно, то в нём можно выбрать максимальный элемент и модуль каждого элемента прообраза будет не больше, чем максимальный элемент X. Тогда прообраз X конечен.
	\section*{Задание 3}
		
	\section*{Задание 4}
		Пусть существуют такие 2 деерва. Так как диаметр каждого из них рааен $d$, то в каждом из них есть путь длины $d$. Добавим между ними ребро так, чтобы диаметр полученного графа стал равен $d$. Для каждой из двух половин этого графа рассмотрим пути от принадлежащей этой половине графа вершины этого ребра до концов путей длины $d$. Каждый из этих путей состоит из пути от вершиныи общего ребра до пути длины $d$ и из части пути длины $d$. Тогда сумма длин этих двух путей до концов пути длины $d$ больше либо равна $d$, так как состоит из длины пути $d$ и удвоенной длины до этого пути, что значит, что хотя бы один из двух путей не короче чем $d/2$. Возьмём эти пути для каждой из двух половин графа и ребро между половинами. Полученный путь будет иметь длину $d+1$ - противоречие.
		Ответ: не сучествует.
	\section*{Задание 5}
		Будем красить вершины по очереди. На каждой вершине смотрим, какие соседи уже раскрашены, смотрим, какие там цвета, так как соседей не более $d$, то они раскрашены не больее, чем в $d$ цветов, то есть есть хотя бы 1 цвет, в который не покрашен никто из соседей. В этот цвет красим рассматриваемую вершину. Если до покраски вершины не было ребра с вершинами одного цвета, то и после покраски вершины не будет. Изначально их небыло, значит, и в конце их не будет и раскраска будет правильной.
	\section*{Задание 8А}
		Если $n$ делится на 2, то при соединении противоположных вершин образуется цикл $v_1$, $v_2$, ..., $v_n1$, $v_1$ длины $n+1$, то есть цикл нечётной длины. Из чего следует, что граф не 2-раскрашиваемый.\par
		Если $n$ не делится на 2, то раскрасим граф по кругу в 2 цвета чередуя цвета. Рёбра, соединяющие противоположные вершины будут соединять вершины разного цвета и раскраска будет правильной.
		\par\includegraphics[width=400px]{~/Downloads/4.png}\par
		Ответ: при $n$ не делится на 2.
	\section*{Задание 8Б}
		Если $n$ делится на 2, то раскрасим граф по кругу в 2 цвета чередуя цвета. После этого, поменяем (среди этих 2 цветов) цвета у всех вершин с номерами большими $n$, а вершины $n+1$ и $2n$ покрасим в 3 цвет. Рёбра, соединяющие противоположные вершины будут соединять вершины разного цвета и раскраска будет правильной.
		\par\includegraphics[width=400px]{~/Downloads/5.png}\par
		Если $n$ не делится на 2, то раскрасим граф по кругу в 2 цвета чередуя цвета, а вершины $n+1$ и $2n$ покрасим в 3 цвет. Рёбра, соединяющие противоположные вершины будут соединять вершины разного цвета и раскраска будет правильной.
		\par\includegraphics[width=400px]{~/Downloads/6.png}\par (точка отмеченная $n$ на самом деле $n-1$)
		Ответ: при $n$ не делится на 2.
\end{document}
