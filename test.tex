\documentclass[a4paper, 12pt]{article}
\usepackage[a4paper,top=1.5cm, bottom=1.5cm, left=1cm, right=1cm]{geometry}
\usepackage[utf8]{inputenc}
\usepackage{mathtext}
\usepackage{amsmath}
\usepackage{amssymb}
\usepackage{amsfonts}
\usepackage{textcomp}
\usepackage[english, russian]{babel}
\usepackage{indentfirst}
\usepackage{fixltx2e}
\usepackage{longtable}
\usepackage{graphicx}
\graphicspath{{pictures/}}
\DeclareGraphicsExtensions{.pdf,.png,.jpg}
\usepackage{natbib}
\usepackage{mathrsfs}
\usepackage[europeanresistors, americaninductors]{circuitikz}

\title{АЛКТГ дз 2}
\author{Татаринов Георгий}
\date{\today}

\begin{document}
	\maketitle
	\section*{Задание 1}
        Предположим, что существует граф, в котором 8 вершин, 23 ребра и есть вершина степени 1.\par
        Тогда так как всего рёбер 23, то суммарная степень всех вершин равна 46. При этом, степень каждой вершины не более 7, при этом степень только одной вершины может быть равна 7, иначе будет хотя бы 2 вершины соединённые с той, у которой степень 1. Тогда суммарная степень всех вершин не более 1+7+6*6 что равно 44 и меньше, чем 46. Противоречие\par
        Ответ: не существует.
	\section*{Задание 2}
		Пусть можем добраться из $A$ в $B$. Тогда число A*10+B делится на 3. Тогда если $A$ делится на 3, то $B$ тоже делится на 3, а если $A$ не делится на 3, то $B$ тоже не делится на 3. Число 1 не делится на 3, а значит, что из него возможно добраться только в те города, номера которых не делится на 3.Тогда добраться в город 9, который делися на 3 невозможно.\par
		Ответ: не возможно
	\section*{Задание 3}
        Если в графе есть путь длины 3, то для того, чтобы крайние рёбра этого пути имели общую вершину, первая и послендняя вершина этого пути будет совпадать, то есть это будет цикл длины 3. Если в таком графе будут ещё рёбра помимо этого цикла, то каждое из этих рёбер будет иметь ровно 1 общую вершину с циклом, а значит, ребро между двумя другими вершинами цикла не будет иметь с ним общих вершин.\par
        Если в граве есть путь длины хотя бы 4, то пусть первые 5 вершин этого пути это $A$, $B$, $C$, $D$, $E$. Тогда если $A=D$, то в этом графе есть и путь длины 3, а значит, весь граф это цикл длины 3, если $A\neq D$, то пути $AB$ и $CD$ не имеют общих вершин - противоречие.
        Если путей длины хотя бы 3 нет, то это граф звезда. Тогда у каждой пары рёбер есть общая вершина.\par
        Ответ:все звёзды и треугольник.
	\section*{Задание 5}
	    Пусть $G$ - Граф из вершин $A$, $B$, $C$, $D$ и рёбер $A-B$, $B-C$, $C-D$, $D-A$,\par
	    $H_1$ - Граф из вершин $A$, $B$, $C$ и рёбер $A-B$, $B-C$,\par
	    $H_2$ - Граф из вершин $C$, $D$, $A$ и рёбра $C-D$, $D-A$.\par
	    Тогда $H_1\bigcap H_2\neq \varnothing$ и при этом граф  $H_1\bigcap H_2$ несвязен.
        \includegraphics[width=100px]{~/Downloads/2.png}
        Ответ: неверно, что граф всегда связен.
	\section*{Задание 6}
	    Пусть $s(x)$ - множество городов, состоящее из $x$ и городов, связанных с $x$.\par
	    Тогда для всех $x$ выполнено $|s(x)|\geq8$\par
	    Пусть $A$ и $B$ - 2 произвольных города. Тогда множества $s(A)$ и $s(B)$ имеют мощность хотя бы 8 и являются подмножествами множества всех городов, которых 15. Так как $8+8=16>15$ а значит, эти множества пересекаются. Тогда до общего города этих двух множеств можно добраться из $A$ и из $B$, а значит, и есть путь между $A$ и $B$.
	\section*{Задание 7}
		Доказать, что в произвольном граве есть 2 вершины одной степени.\par
		Доказательство: пусть в графе $n$ вершин. Тогда степени вершин могут принимать значения от $0$ до $n-1$. При этом, если у вершины степень равна $n-1$, то она соединена со всеми остальными, а если степень равна $0$, то вершина не связана ни с чем. То есть в графе не могут одновремено находиться вершины степени $0$ и $n-1$. Тогда степени вершин в графе могут принимать значения от $0$ до $n-2$ или от $1$ до $n-1$. В обоих случаях набор значений для степеней вершин меньше, чем количество вершин, а значит, найдутся вершины одной степени.
	\section*{Задание 8}
		Если в графе на $n$ вершинах есть $k$ рёбер, то в дополнении будет $C_n^2-k$ рёбер. При этом, в графе-пути на $n$ вершинах $n-1$ рёбер, а в графе-цикле на $n$ вершинах $n$ рёбер. Поэтому, $n-1\leq k\leq n$ и $n-1\leq C_n^2-k\leq n$. Тогда $2*n-2\leq C_n^2\leq 2*n$ из чего следует, что $n=1$ или $n=4$ или $n=5$. И все эти $n$ подходят.\par \par \par
        \includegraphics[width=500px]{~/Downloads/3.png}
\end{document}
